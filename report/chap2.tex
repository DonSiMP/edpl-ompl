\chapter{Issues \& Notes}

\section{Notes}

OMPL’s API is very different programmatically and philosophically from PMPL. Due to its lightweight nature, OMPL does not depend 
external libraries (except BOOST) which gives it a disadvantage in terms of extra functionality (such as reading setup parameters from XML files etc.).

\subsection{How the OMPL PRM planner works}
\begin{enumerate}
 \item Call \textit{setProblemDefinition} to define the parameters of the problem.
 \item Call \textit{setup} to setup the planners specific internals like validity checker etc.
 \item  Call \textit{solve} to find construct graph and find a solutions:
  \begin{enumerate}
    \item Call \textit{checkValidity} : makes sure start and goal states have been set
    \item set termination conditions
    \item Call \textit{constructRoadMap}: 
	    \begin{itemize}
	     \item Allocates a simple state sampler
	     \item Calls \textit{growRoadmap} and \textit{expandRoadmap} in a 2:1 ratio
	    \end{itemize}
  \end{enumerate}

 \item \textit{addMilestone} : 
  \begin{enumerate}
   \item  Adds a state to the graph and creates a vertex with the properties and everything else. 
   \item After creating the vertex, it attempts to connect the vertex to its nearest neighbours
   \item during the connection, it finds the motion cost associated with the connection. \color{red}{This is the point where we run the monte carlo simulation to generate edge costs}
  \end{enumerate}

 \item \textit{growRoadmap}: Takes as input a termination condition and a state, samples a valid state and puts it in input state and adds it to the graph.
\end{enumerate}

\subsection{Sampling a valid belief}

\begin{enumerate}
 \item While sampling we need to check if a state is in collision and observable. 
 \item The collision check is done by validitychecker. 
 \item The observability check can be done by filter or observation model (currently chosen).
 \item If sampled state passes all checks, we need to construct its stationary belief.
\end{enumerate}


\subsection{State Validity Checking}
Can code be simply re-used, is there any need to write a new validitychecker? 

\begin{itemize}
 \item We need to check for collision. Can use the provided FCL wrapper in OMPL.
 \item The observability check can also done through validitychecker because the information about space should be accessed through the space information.

\end{itemize}

\subsection{SpaceInformation Class}

We use the \textbf{ompl::control::SpaceInformation} class. This combines the statespace and controlspace information. This requires us to set:
\begin{enumerate}
 \item ValidityChecker: Checks if a state is collision free.
 \item MotionChecker: Checks if we can have a valid motion (collision free and satisfying dynamic constraints) between two states.
 \item Sampler: Samples a valid state which can be added to the graph as a node.
 \item StatePropagator: Given a control and its duration, it propagates the system.
\end{enumerate}

\subsection{The OMPL MotionValidator class}

\begin{enumerate}
 \item Checks if the intermediate states between start and end states are valid. Is basically the localplanner.
 \item Should we add the monte carlo simulation to this class or keep our own MotionModelLP to do that.
\end{enumerate}

\subsection{Using the Control Space}

\begin{itemize}
 \item OMPL defines a control space and we must use RealVectorControl instead of armadillo (old pmpl implementation). So need to change how we use \textcolor{red}{ControlType}. Instead of colvec it should use the OMPL stuff. 
 
 \item An instance of a \textbf{StatePropagator} class is to be defined in controlspace. Its job is to propagate the state of a robot based on control input. We need to  derive      our motion model from this class. 
 
  \item We can use the \textbf{SpaceInformation} class belonging to control space for our planner. \textcolor{red}{Should we derive a space from control:spaceinformation and then provide it access to the motion model and observation model? Since the spaceinformationptr gets passed around to all important classes, this will ease implementation.} 

 \end{itemize}

 \subsubsection{How to define an Edge}
 In PMPL, we used a Controller class to describe our edges [used it for node controllers also]. These controllers (edges) were stored in a vector and 
 their indexes were stored for each edge. OMPL gives us a \textbf{PathControl} class, which defines a path in control space. It can be used to derive a new
 FIRMEdge class.

\subsection{Modifications required to OMPL PRM}

\begin{enumerate}

 \item Check if a state is observable before adding it to the graph. The \textbf{UniformValidBeliefSampler} and   \textbf{GaussianValidBeliefSampler} will check the observability for a sample, whether to actually put the observability check in filter or observation model needs to be decided
   
 \item In the function \textit{addMilestone}:
  \begin{enumerate}
   \item Need to calculate the cost-to-go between vertices. This is the edge cost + stabilization cost. Should we define it as an optimization objective ?
  \end{enumerate}

  \item Should the motionmodel and observationmodel be provided to the planner or to problemdefinition (like pmpl) or to a new spaceinformation ?
\end{enumerate}

We can create the controller class inside the planner. Would make sense.

\subsubsection{Calculating Cost-To-Go}

We can define a new \textbf{InformationOptimizationObjective} whose job is to calculate the cost-to-go. This way we can minimize the changes required to the PRM
class.

\subsection{Find the solution path}

\subsubsection{What was happening in ompl::PRM}
Function \textit{constructSolution} calls the Boost A* search to find the list of nodes. Then it calls \textit{constructGeometricPath} to add the list of nodes
to a geometric path.


\subsubsection{What we need to do in FIRM}

\begin{itemize}
 \item Create a new \textbf{FeedbackPath} class
 \item Add a DP solver to FIRM, which tells you the feedback path
 \item Will need to add 
\end{itemize}


\section{Issues}
Log of issues faced and discussions on their solutions.

\begin{enumerate}
 \item Should the optimization functionality in OMPL Planner class be used or should the planner be implemented straight forward, similar to PMPL approach ?
    \begin{itemize}
     \item 
    \end{itemize}
\item How to add the properties of an edge to the graph i.e. EdgeController; a) Create a hashtable which maps from edge to controller id, b) store the edge controller as the property of an edge i.e. create a custom edge property. Easier approach would be to (a).

 \item What is the job of the validitychecker ? -> to check if a state is in collision or not. 

\end{enumerate}
